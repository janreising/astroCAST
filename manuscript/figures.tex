\begin{figure}[h!]
\begin{center}
\includegraphics[width=\linewidth]{figures/1.png}
\end{center}
\caption{AstroCAST Pipeline Overview: First box shows how the input video is converted and processed to detect events. Second box shows how the detected events undergo feature extraction utilizing different approaches to generate feature vectors. Last steps shows how feature vectors can be analyzed to identify patterns and correlations. Optional steps in the pipeline are indicated by dotted boxes, and outputs at each stage are shown in blue. Exemplary types of experiments explained in this protocol are highlighted in red.}\label{fig:1}
\end{figure}

\begin{figure}[h!]
\begin{center}
\includegraphics[width=\linewidth]{figures/2.png}
\end{center}
\caption{Screenshot of the \ac{GUI} displaying the converted video file of astrocytic calcium fluorescence (\ref{Tbl1}). The video has been downsampled and captured at a 20X magnification, focusing on the \ac{preBötC} region.}\label{fig:2}
\end{figure}

\begin{figure}[h!]
\begin{center}
\includegraphics[width=\linewidth]{figures/3.png}
\end{center}
\caption{Comparative Visualization of Denoising on Video Frames and Pixel Intensity Measurements. Single frame (512x512 pixels) before denoising (left panel), showing notable background, and the corresponding frame after the application of a denoising algorithm, showing reduced background and enhanced clarity (middle panel). Pixel intensity traces from four selected points (right panel, P0-P3) in the video frame before denoising, displaying the variation over 200 frames. Right Plot) Pixel intensity traces for the same points after denoising, indicating a more stable intensity profile, while retaining features of the signal. The denoising algorithm was applied to a (128, 128) field of view with a configuration of 5 frames, each bordering the target frame, and no gap frames. The denoiser, trained for 50 epochs on a custom dataset, had a learning rate of 0.0001, momentum of 0.9, 3 layer stacks (n\_stacks), and 64 kernels of size 3 in the first layer without batch normalization. Inference involved a 10-pixel overlap in each direction with 'edge' padding. The test and validation were conducted using the same custom training dataset.
}\label{fig:3}
\end{figure}

\begin{figure}[h!]
\begin{center}
\includegraphics[width=\linewidth]{figures/4.png}
\end{center}
\caption{Background Subtraction in Fluorescence Imaging Analysis. Comparative analysis of fluorescence images and pixel intensity traces before (top left) and after background subtraction (bottom left). Pixel intensity over time (right panels). These plots reveal the background noise reduction, where post-subtraction y-axis values are near zero, indicating effective subtraction. Event shapes remain consistent, verifying reliable preservation of events. Notably, the second panel (green) slightly alters the signal strength of events, underscoring the importance of careful parameter optimization or exclusion of this step if it compromises data fidelity. Please refer to Video S2\ref{} for a dynamic visualization.}\label{fig:4}
\end{figure}

\begin{figure}[h!]
\begin{center}
\includegraphics[width=\linewidth]{figures/5.png}
\end{center}
\caption{Overview of Event Detection in Calcium Fluorescence Imaging. A) Calcium Fluorescence Processing: Post-motion correction and denoising, showcasing calcium fluorescence. Red dots signify the selected pixels for detailed analysis in Panel E. B) Event Identification: Events detected through both temporal and spatial thresholding techniques. Colored regions highlight the active events within specific pixels, illustrating the spatial distribution of neuronal activity. C) Temporal Thresholding: A binary mask generated by applying a temporal threshold, delineating periods of significant activity versus inactivity. D) Spatial Thresholding: A binary mask created by spatial thresholding to identify active regions, emphasizing the spatial aspects of neuronal activity. E) Pixel Intensity Analysis: Examination of pixel intensities for pixels indicated in Panel A. Colored regions within the plot correspond to frames identified as active, providing a temporal resolution of neuronal activity.}\label{fig:5}
\end{figure}

