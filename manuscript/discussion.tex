\todo{Athina}{suggestions for updated discussion section}{I am not sure how to write this section given the updated manuscript. I would be grateful for some ideas.}

\subsection{Expected outcomes}

AstroCAST could facilitate several key outcomes for researchers working with astrocytic calcium imaging data analysis
. On the technical front, the tool is designed to handle large-scale datasets effectively. For example, in time
-lapse recordings of astrocytes at 20X magnification and 1024x1024 pixel resolution, astroCAST is equipped to
identify up to 1,000 astrocytic events per 100 frames. As the analysis progresses, users should find that the
software can comfortably manage up to 50,000 events using the encoding algorithms provided. This performance equates
to the comparative analysis of around 15 time-lapse videos, each with a 20-minute recording time. Furthermore,
astroCAST is designed to efficiently handle larger datasets, efficiently analyzing more events than previously
possible in astrocyte-specific studies. This is achieved without compromising on computational time. From an
innovation standpoint, astroCAST offers more than just event detection. Users can expect a cohesive, end-to-end
analytical workflow that guides them from raw video data to insightful experimental conclusions. The software brings
structure and flexibility to the often-challenging variable-length timeseries analysis, setting it apart as a unique
tool in the field of astrocytic calcium imaging.

\subsection{Limitations}

The astroCAST protocol has proved to be effective in analyzing calcium events in fluorescence imaging data. We have
tested the system in depth on acute slices, as well as available public data for in-vivo and cell culture recordings
. There are, however, challenges that might be considered as limitations. For example, astroCAST can firmly attribute
events to individual cells, but is agnostic to cell shape by design, enabling us to analyze recordings even when
cells overlap or change morphology. Furthermore, issues that stem from the difficulty to distinguish close bordering
events, the computational complexity, the event length and data dimensionality, and others, may also need fine-tuning
. We acknowledge that users might encounter some of these, and we make suggestions for each in the troubleshooting
section.

We consider the fact that astroCAST is user customizable to be an asset that will aid researchers globally to
decipher the enigmatic astrocytic calcium activity.