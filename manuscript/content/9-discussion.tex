\todo{Eric}{which references are missing}{Eric could you pleaes point out the sentences that you are lacking references for?}

\ccomment{Eric}{In the field of neurobiology}{Overall again, a bit of ChatGTP feel on tgjis..- True or jut me . Overall SHORTER sentences. will benefit readability and to get the message that manuscript tries to convey through better. \newline Not a rambling of data. Higlight the crucial thing in a short initial paragraph then folloed by HOW this tol has pros and cons) versus esisting with dequated references!}
, it is important to acknowledge the substantial gap in open-source, customizable tools tailored for the comprehensive characterization and dynamics of astrocytes. This scarcity significantly hinders the advancement of our understanding of the complex functions carried out by these glial cells. Their function extends far beyond their traditional role of supporting neuronal activity\citep{montalant_role_2021,ransom_new_2003}. Thus, the development and refinement of specialized tools such as astroCAST represents a step forward in addressing this limitation. Unlike previous methodologies that regard astrocytes as mere adjuncts to neuronal studies, astroCAST aimed to meticulously and explicitly address the uniqueness of astrocyte research\tref{transposed-comparison}.

AstroCAST offers comprehensive options for analyzing and interpreting activity events repurposing existing tools from various data science domains for astrocyte research. This approach highlights the importance of leveraging accumulated knowledge across fields to enhance the efficiency and breadth of astrocyte studies. Moreover, the advances in bioimaging dataset processing are still lagging behind other big data analysis approaches. Addressing this gap, astroCAST is a dedicated ML-driven toolkit that can be used to identify, denoise and characterize the dynamics of a set cell population.

\todo{Athina}{proof read the following paragraph}{I added the next paragraph to discuss the advantages and disadvantages of the different modules available to analyze experiments. Could you have a general proof read? I tried to smooth out the transition between the previous and following paragraph, but am not sure how well I succeeded. Thank you :) }
In evaluating the effectiveness of different embedding and analysis approaches for astrocytic calcium imaging data, we assessed how each method handles the intricate dynamics of astrocytic events, focusing on their performance across various criteria such as accuracy, generalization to new data, and computational efficiency\fref{6}. When provided with a dataset containing two groups with identical characteristics, all methods could perfectly memorize the training dataset, but failed to generalize this knowledge to unseen data. This outcome aligns with our expectations, as astrocytic events of identical shapes should indeed be indistinguishable.
Feature extraction (FExt) emerges as a quick and straightforward method, albeit less effective with complex event types. Convolutional Neural Networks (CNNs) offer speed and accuracy but falter with astrocytic events' intrinsic variable lengths. Attempts to enforce fixed lengths or extend events risk altering their inherent shapes and producing spurious results. Recurrent Neural Networks (RNNs) show promise in dealing with variable event lengths but are challenging to fine-tune, as evidenced by their subpar performance in the triplet condition (accuracy of 0.56). Pearson correlation analysis proved wholly inadequate, with all clustering algorithms failing to differentiate between events with significant variances (0.5 prediction accuracy), except for the trivial case involving variable lengths. Dynamic Time Warping (DTW) excels in handling variable lengths but is slower and slightly less accurate (0.94 accuracy compared to others' 0.99) for fixed-length events. However, its avoidance of an embedding step compensates for the slower processing speed, albeit with a significant memory constraint as the number of events increases (O($n^2$) complexity).
In summary, while each approach has its strengths and weaknesses in processing astrocytic calcium imaging data, our findings highlight the crucial balance between accuracy, adaptability to variable event lengths, and computational demands.

The intended audience of astroCAST primarily comprise bioinformaticians or neuroscientists with a solid foundation in programming. Proficiency in Python and familiarity with executing command line tools are essential prerequisites for effectively utilizing this   toolkit. Importantly, troubleshooting code is a necessary skill set for circumnavigating potential challenges during data processing and personalizing experimental implementation. Prior experience in image analysis, time series clustering, and machine learning is advantageous but not mandatory. However, to decode the intricacies of astrocytes, it is essential employ a multidisciplinary team skilled in both experimental neuroscience and computational analysis.

In our experience, astroCAST's \ac{GUI} responsiveness was a key limitation, indicating an area of enhancement for future versions. Furthermore, the toolkit could not pass the threshold of "inference-of-cell" implementation, which would assign events to individual cells. Thus, the "Functional Units" module currently exists as an experimental feature, but it lacks proper validation and testing due to insufficient biological data (e.g., sparse Gcampf6 expression recordings or post-imaging staining).

However, AstroCAST can facilitate several key outcomes for researchers working with astrocytic calcium imaging data analysis. On the technical front, the tool is designed to handle large-scale datasets effectively. For  example, in time-lapse recordings of astrocytes at 20X magnification and 1024x1024 pixel resolution, astroCAST is equipped to identify up to 1,000 astrocytic events per 100 frames. As the analysis progresses, users should find that the software can comfortably manage up to 100,000 events using the algorithms provided. This performance equates to the comparative analysis of around 7 time-lapse videos, each with a 20-minute recording time.

The astroCAST toolkit has proved to be effective in analyzing calcium events in fluorescence imaging data. We have tested the system in depth on acute slices, as well as available public data for in-vivo and cell  culture recordings. There are, however, challenges that might be considered as limitations. For example, due to its design agnosticism towards cell shape, astroCAST cannot firmly attribute events to individual cell. This allows for the analysis of recordings even when cells overlap or change morphology. Furthermore, issues that originate from the difficulty to distinguish close bordering events, the computational complexity, the event length and data dimensionality, and others, may also need fine-tuning. We acknowledge that users might encounter some of these, and we make suggestions for each in the troubleshooting section. We regard astroCAST's flexible modular design as crucial for overcoming challenges, empowering researchers to translate astrocytic calcium signalling into biological insights.

AstroCAST is designed to handle larger datasets, analyzing more events than previously possible in astrocyte-specific studies with enhanced efficiency. This is achieved without compromising computational time. From an innovation standpoint, astroCAST offers more than just event detection. Users can expect a cohesive, end-to-end analytical workflow that guides them from raw video data to insightful experimental conclusions. The software brings structure and flexibility to the often-challenging variable-length timeseries analysis, setting it apart as a unique tool in the field of astrocytic calcium imaging. AstroCAST not only enhances our ability to probe astrocyte activity, but also holds the promise of unveiling previously unexplored aspects of their role in neural circuitry and brain development and function. This demonstrates the potential for future synergistic integration of advanced imaging techniques, such as calcium imaging, with established and high-resolution methods like immunohistochemistry or spatial transcriptomics, offering a wealth of new possibilities for in-depth astrocytic analysis.