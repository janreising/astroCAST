
%%%%%%%%%%%%%%%%%%%%%%%%%%%%%%%%%%%
% Table 1: functionality comparison

\bgroup
\def\arraystretch{1.5}
\begin{table}[htb]
    \centering
    \caption{Availability of different functionalities of astroCAST across
    operating systems. A check mark (\cmark) denotes full compatibility and active support. An asterisk (\optional)
        indicates that the
    functionality is optional and can be installed upon user request (see below). A cross mark (\xmark) signifies
        that the
    functionality is not supported or available on the operating system. A half bullet (\halfbullet) indicates
        partial support,
    indicating that the functionality is expected to be available but is not actively tested. \newline}
    \label{tab:functionalities}
    \begin{tabular}{|l|c|c|c|c|}
        \hline
        \textbf{Functionality} & \textbf{Linux} & \textbf{Windows} & \textbf{MacOS (Intel)} & \textbf{Docker} \\ \hline
        Preprocessing          & \cmark         & \cmark           & \cmark                 & \cmark          \\ \hline
        Motion correction      & \cmark         & \cmark           & \cmark                 & \cmark          \\ \hline
        Denoising              & \cmark         & \halfbullet      & \cmark                 & \cmark          \\ \hline
        Delta                  & \cmark         & \cmark           & \cmark                 & \cmark          \\ \hline
        Detection              & \cmark         & \cmark           & \cmark                 & \cmark          \\ \hline
        UMAP outlier detection & \cmark         & \cmark           & \xmark                 & \cmark          \\ \hline
        Encoding - all         & \cmark         & \cmark           & \cmark                 & \cmark          \\ \hline
        Analysis               &                &                  &                        &                 \\ \hline
        DTW distance           & \cmark         & \cmark           & \xmark                 & \cmark          \\ \hline
        Barycenters            & \cmark         & \cmark           & \xmark                 & \cmark          \\ \hline
        Video player \optional & \cmark         & \cmark           & \cmark                 & \xmark          \\ \hline
    \end{tabular}
\end{table}
\egroup

%%%%%%%%%%%%%%%%%%%
% Table 2: Datasets

\bgroup
\def\arraystretch{1.5}
\begin{table}[htb]
    \centering
    \caption{Overview of Datasets Employed in Analysis. \newline}
    \label{tab:datasets}
    \begin{tabular}{|l|l|l|l|l|l|}
        \hline
        \textbf{Dataset} & \textbf{Publication} & \textbf{Experiment} & \textbf{Imaging} \\ \hline
        train/test & previously unpublished & acute slices, Gcamp6f & \ac{1P} (8 Hz) \\ \hline
        ExVivo & AQuA\citep{wang_event-based_2018} & acute slices, Gcamp6f & \ac{2P} (1.1 Hz) \\ \hline
        Glusnfr & AQuA\citep{wang_event-based_2018} & acute slices, GluSnFR & \ac{2P} (4-100 Hz) \\ \hline
        InVivo & AQuA\citep{wang_event-based_2018} & in vivo GCamp6f & \ac{2P} (2 Hz) \\ \hline
        cellscan\_scim & CHIPS\citep{barrett_chips_2018} & blood vessels & \ac{2P} \\ \hline
    \end{tabular}
\end{table}
\egroup

%%%%%%%%%%%%%%%%%%%%
% Table 3: embedding

\bgroup
\def\arraystretch{1.5}
\begin{table}[htb]
    \centering
    \caption{Evaluation of different methodologies for Embedding and Distance calculations. Highlights the requirements and limitations. A (\cmark) indicates a necessary requirement, items with (\xmark) are optional and a (\halfbullet) indicates a difference between operating systems.\newline}
    \label{tab:embedding}
    \begin{tabular}{|l|c|c|c|c|c|}
        \hline
        & \textbf{Feature} & \textbf{Convolutional} & \textbf{Recurrent} & \textbf{Pearson} & \textbf{Dynamic Time} \\
        & \textbf{Extraction} & \textbf{Autoencoder} & \textbf{Autoencoder} & \textbf{Correlation} & \textbf{Warping} \\ \hline
        Abbreviation & \textbf{FExt} & \textbf{CNN} & \textbf{RNN} & \textbf{pearson} & \textbf{DTW} \\ \hline
        Fixed length & \xmark & \cmark & \xmark & \xmark & \xmark \\ \hline
        Normalization & \xmark & \cmark & \cmark & \xmark & \xmark \\ \hline
        Training & \xmark & \cmark & \cmark & \xmark & \xmark \\ \hline
        Memory limited & \xmark & \xmark & \xmark & \cmark & \cmark \\ \hline
        Wall time limited & \cmark & \xmark & \xmark & \cmark & \cmark \\ \hline
        Parallelized & \xmark & \xmark & \xmark & \cmark & \halfbullet \\ \hline
    \end{tabular}
\end{table}
\egroup

%%%%%%%%%%%%%%%%%%%%%%
% Table 4: Experiments

\bgroup
\def\arraystretch{1.5}
\begin{table}[htb]
    \centering
    \caption{Comprehensive overview of experiment types that are included in astroCAST, with their respective requirements.\newline}
    \label{tab:experiments}
    \begin{tabular}{|l|c|c|c|c|}
        \hline
        & \multicolumn{2}{c|}{\textbf{Conditional Contrasts}} & \multicolumn{2}{c|}{\textbf{Coincidence Detection}} \\ \hline
        & \textbf{Classifier} & \textbf{Hierarchical} & \textbf{Classifier} & \textbf{Regression} \\ \hline
        Training required & \cmark & \xmark & \cmark & \cmark \\ \hline
        User labels & group & group & timing & timing \\ \hline
        Input type & embedding & distance & embedding & embedding \\ \hline
    \end{tabular}
\end{table}
\egroup