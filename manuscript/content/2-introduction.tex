% background
Analyzing cytosolic calcium oscillations helps decode neuronal firing patterns, synaptic activity, and network dynamics, offering insights into cell activity and states\citep{del_negro_sodium_2005,grienberger_imaging_2012,dombeck_imaging_2007}. Furthermore, the changes in calcium activity may be indicative of cell responses to downregulation of molecular pathways, epigenetic alterations, or even the effect of treatments or drugs, and disease states\citep{lines_astrocytes_2020,miller_calcium_2023,robil_glioblastoma_2015,huang_vitro_2013,britti_tau_2020,zhang_estrogen_2010}. This makes dynamic calcium activity recordings a crucial tool to investigate physiology and neurological disorders and to design and develop therapeutic interventions.

% problem statement
While we have seen major development in recent years in the imaging tools available to study astrocytes, the software side has been slow to catch up\citep{gorzo_photonics_2022,aryal_er-gcamp6f_2022,stobart_cortical_2018}. Currently several software packages are available for researchers to analyze calcium activity recordings from brain cells\tref{transposed-comparison}. However, most of these packages are primarily developed for neurons, and are often not suitable for astrocytes. The challenge stems from astrocytes' unique physiology, marked by rapid subcompartment calcium fluctuations\citep{stobart_long-term_2018,curreli_complementary_2022} and their ability to alter the morphology during a single recording\citep{anders_epileptic_2024,baorto_astrocyte_1992}. Astrocytes exhibit calcium fluctuations that are spatially and temporally diverse, reflecting their integration  of a wide range of physiological signals\citep{semyanov_making_2020,smedler_frequency_2014,denizot_simulation_2019,papouin_astrocytic_2017,bazargani_astrocyte_2016}.

The available toolkits specifically transcribing astrocytic activity come with several shortcomings\tref{transposed-comparison}. A major barrier is the use of proprietary programming languages like Matlab, that hinder widespread use. Additionally, slow implementations due to lack of parallelization and \ac{GPU} acceleration or usage of RAM exceeding most standard setups impedes analyze of long videos (>5000 frames). While all compared toolkits offer efficient event detection, most lack support to gain insight from the extracted events.

% the solution (AstroCAST)
AstroCAST, developed in Python for its versatility and ease of use, addresses these shortcomings through a modular design, allowing for customizable pipelines and parallel processing.  It optimizes resource use by only loading data as needed, ensuring efficient hardware scaling. Its modular design enables stepwise quality control and flexible customization. Finally, astroCAST includes dedicated modules to analyze common research questions, as compared to other packages that primarily focus on event detection.

% emphasize novelty and impact
Here, we not only introduce astroCAST but also provide a detailed guide for extracting astrocytic calcium signals from video data, performing advanced clustering and correlating their activity with other physiological signals\fref{1}. AstroCAST extends beyond theory, having been tested with synthetic data, as well as in-vitro, ex-vivo and in-vivo recordings\tref{datasets}. This underscores AstroCAST’s ability to harness sophisticated computational techniques, making significant strides in the study of astrocytes and offering new avenues for neuroscience research.
