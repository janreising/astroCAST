\subsection{Event Detection}
%Timing: 1 min – 3h
%Running time will increase linearly with the number of events in the recording. 

AstroCAST employs an event-centric approach to event detection, adapted from the AQuA package\citep{wang_event
-based_2018}. Building on the AquA algorithm, astroCAST incorporates both spatial and temporal thresholding
techniques to identify and extract events effectively. Spatial thresholding is executed based on either a single
frame or a small volume, facilitating the identification of significant patterns or features in space. Concurrently,
temporal thresholding is implemented by scrutinizing individual pixels for peaks across the time dimension,
essentially tracking fluorescence changes over time.

Following the thresholding process, a binary mask is generated, which is used to label connected pixels, considering
the full 3D volume of the video. This allows for the integration of different frames in the z-dimension to constitute
a single event in a 3D context, ensuring a comprehensive analysis of the events captured in the video. Upon
successful event detection, our protocol generates a folder with a .roi extension. This folder houses all files
related to the detected events, offering an organized and accessible repository for the extracted data.

The event detection step is the most critical aspect of astroCAST and is sensitive to the choice of parameters. We
have therefore implemented an interactive interface to explore the performance of each step during event detection
given the relevant options. The program will prompt a link to the browser-based interface. We recommend to select a
short range of frames in the video file to keep processing time during exploration short.

\begin{lstlisting}[style=bashStyle]
    $ astrocast explorer -- input path "/path/to/file" -- h5-loc "/dataset/name"
\end{lstlisting}

Once users are satisfied with the chosen parameters, they can be exported to a YAML config file, that can be provided
to the astrocast \ac{CLI}.

\subsubsection{Detection Module Description}
The detection module in AstroCAST applies a comprehensive approach to identify astrocytic events by smoothing,
thresholding, and applying morphological operations on fluorescence imaging data. This section outlines the key steps
and parameters used in the process.

\paragraph{Smoothing}
AstroCAST incorporates a Gaussian smoothing kernel to enhance event detection while preserving spatial features.
Adjusting the \lstinline[style=bashStyle]{--sigma} and \lstinline[style=bashStyle]{--radius} parameters allows for
the refinement of the smoothing effect, ensuring that events are emphasized without compromising the integrity of
spatial characteristics. The smoothed data then serve as the basis for subsequent detection steps, although users can
opt out of smoothing if desired.

\paragraph{Thresholding}
\todo{Jan}{Active pixels}{thresholding -> identifying active pixels}
Thresholding in AstroCAST is executed in two phases, spatial and temporal, to classify pixels as active with high
precision.

Spatial thresholding evaluates the entire frame to distinguish active pixels, automatically determining a cutoff
value. It incorporates the mean fluorescence ratio of active to inactive pixels to mitigate the misidentification of
noise as events. The \lstinline[style=bashStyle]{--min-ratio} parameter sets the minimum ratio threshold, and the \lstinline[style=bashStyle]{--z-depth} parameter allows for the consideration of multiple frames to improve threshold accuracy.

Temporal thresholding analyzes the video as a series of 1D time series, identifying peaks with a prominence above a
user-defined threshold. Parameters such as \lstinline[style=bashStyle]{--prominence}, \lstinline[style=bashStyle]{--width}
, \lstinline[style=bashStyle]{--rel-height}, and \lstinline[style=bashStyle]{--wlen} fine-tune the detection of
events over time, enhancing the separation of true events from noise.

Combining spatial and temporal thresholding, users can choose to merge or differentiate the active pixels identified
by each method, optimizing event detection based on specific dataset characteristics.

\paragraph{Morphological Operations}
To address artifacts resulting from thresholding, such as holes (false negatives) and noise (false positives),
AstroCAST employs morphological operations. Parameters like \lstinline[style=bashStyle]{--area-threshold} and \lstinline[style=bashStyle]{--min-size} adjust the maximum hole size to fill and the minimum size of active pixel clusters, respectively. These operations, which also consider connectivity and the \lstinline[style=bashStyle]{--z-depth} parameter, refine the detection outcome by smoothing gaps and removing minor artifacts, enhancing the clarity and segmentation of detected events.

\paragraph{Additional options}
Additionally, AstroCAST provides the option to exclude the video border from active pixel detection (\lstinline[style=bashStyle]{--exclude-border}) to mitigate motion correction artifacts. An experimental feature (\lstinline[style=bashStyle]{--split-events}) is available for separating incorrectly connected events, further improving the accuracy of event detection.

\todo{Jan}{Figure}{Add figure combining several previous figures: overview???; Fig. 8 (temporal thresholding; Spatial
thresholding??; Fig. 12: Detected events overlay )}

\subsubsection{Evaluating Detection Quality}
AstroCAST offers a streamlined approach for assessing the quality of event detection through its command-line interface.

\begin{lstlisting}[style=bashStyle]
    $ astrocast view-detection-results "/path/to/roi"
\end{lstlisting}

Researchers can visually inspect the detection outcomes. During this evaluation phase, it is crucial to determine
whether the detection process has successfully identified all expected events, thereby minimizing false negatives,
and to assess the correctness of these events to ensure that noise has not been misclassified as true events, which
would indicate false positives. If the detection results are not satisfactory, revising the event detection
parameters or adjusting settings in the preceding Delta calculation step may be necessary to achieve improved
accuracy. For a dynamic visualization that illustrates the impact of parameter adjustments on detection outcomes,
researchers are directed to consult Video S2 \missing{Video S2}.
