\subsection{Preprocessing}

\subsubsection{File conversion}
% Timing: 1-15 min (10M – 10Gpixels)
% Increasing the number of frames increases the processing time approximately linearly.

AstroCAST is designed to handle a wide range of common input formats, such as .tiff and .czi files, accommodating the
diverse nature of imaging data in neuroscience research. One of the key features of astroCAST is its ability to
process interleaved datasets. By specifying the \lstinline[style=bashStyle]{--channels} option, users can
automatically split datasets based on imaging channels, which is particularly useful for experiments involving
multiple channels interleaved, such as alternating wavelengths.

Moreover, astroCAST supports the subtraction of a static background from the video recordings using the \lstinline[
    style=bashStyle]{--subtract-background} option. This feature allows for the provision of a background image or
value, which is then subtracted from the entire video. Subtracting the dark noise of the camera, for instance, can
significantly enhance the quality of the resulting data.

For optimal processing and data management, it is recommended to convert files to the .h5 file format. The .h5 format
benefits from smart chunking, enhancing the efficiency of data retrieval and storage. The output configuration can be
finely tuned with options such as \lstinline[style=bashStyle]{--h5-loc} for specifying the dataset location within
the .h5 file, \lstinline[style=bashStyle]{--compression} for selecting a compression algorithm (e.g., `gzip`), and \lstinline[style=bashStyle]{--dtype} for adjusting the data type if the input differs from the intended storage format.

Chunking is a critical aspect of data management in astroCAST, allowing for the video to be divided into discrete
segments for individual saving and compression. The \lstinline[style=bashStyle]{--chunks} option enables users to
define the chunk size, balancing between retrieval speed and storage efficiency. An appropriately sized chunk, like (
100,100,100), can significantly improve processing speed without compromising on efficiency. In cases where the
optimal chunk size is uncertain, setting \lstinline[style=bashStyle]{--chunks} to `None` allows astroCAST to
automatically determine a suitable chunk size.

Lastly, the \lstinline[style=bashStyle]{--output-path} option directs astroCAST where to save the processed output.
While the .h5 format is recommended for its efficiency, astroCAST also supports saving in formats such as .tiff, .tdb
, and .avi, providing flexibility to accommodate various research needs and downstream analysis requirements.

\begin{lstlisting}[style=bashStyle]
    # (optional) browse the available flags
    $ astrocast convert-input --help
    # '--config' flag can be ommited to use default settings
    $ astrocast -–config "/path/to/config" convert-input "/path/to/file/or/folder"
\end{lstlisting}

Verify the conversion through a quick visual inspection using the built-in data viewer\fref{2} or an imaging software
of your choice (e.g., ImageJ). During this check, ensure that the pixel values are within the expected range (int,
float), the image dimensions (width and height) are as anticipated, all frames have been successfully loaded, de
-interleaving (if applicable) has been executed correctly, and the dataset name is accurate.

\begin{lstlisting}[style=bashStyle]
    $ astrocast view-data --h5-loc "dataset/name" "/path/to/your/output/file"
\end{lstlisting}

\todo{Jan}{add explanation for lazy parameter}{}

\subsubsection{Motion Correction (optional)}
% Timing: 1-15 min (10M – 10Gpixels)
% Increasing the number of frames increases the processing time approximately linearly.

During imaging, samples will drift or warp slightly which means that the location of the imaged astrocytes might
change in the \ac{FOV}. Depending on the required analysis, this can have detrimental consequences to the results.
The motion correction module of the https://github.com/flatironinstitute/CaImAn15, based on implementation in CaImAn\citep{3}, is used in the astroCAST protocol to correct these artifacts. Please see Supplementary Video 1 (S1) \ref{S1} for reference.

We are including the commonly adjusted parameters here. We refer users to the the jnormcorre readme \ref{readme} for
detailed information.

A cornerstone of this feature is the ability to set the maximum allowed rigid shift through the \lstinline[style=
bashStyle]{--max-shifts} parameter, with a default value of 50 pixels in both the x and y dimensions. This parameter
is critical for accommodating sample motion, ensuring that shifts do not exceed half of the image dimensions, thus
balancing between correction effectiveness and computational efficiency.

The precision of motion correction is further refined using the \lstinline[style=bashStyle]{--max-deviation-rigid}
option, which limits the deviation of each image patch from the frame's overall rigid shift to a default of 3 pixels
. This ensures uniformity across the corrected image, enhancing the fidelity of the motion correction process.

For iterative refinement of the motion correction, astroCAST employs the \lstinline[style=bashStyle]{--niter-rig}
parameter, allowing up to three iterations by default. This iterative approach enables a more accurate adjustment to
the motion correction algorithm, improving the quality of the processed images.

To address non-uniform motion across the field of view, astroCAST offers the piecewise-rigid motion correction option
, activated by setting \lstinline[style=bashStyle]{--pw-rigid} to True. This method provides a more nuanced
correction by considering different motion patterns within different image segments, leading to superior correction
outcomes.

Furthermore, the \lstinline[style=bashStyle]{--gsig-filt} parameter anticipates the half-size of cells in pixels,
with a default setting of 20 pixels. This information aids in the filtering process, crucial for identifying and
correcting motion artifacts accurately.

By carefully adjusting these parameters, researchers can significantly enhance the quality of their imaging data,
ensuring that motion artifacts are minimized and that the resultant data is of the highest possible fidelity for
subsequent analysis.

\begin{lstlisting}[style=bashStyle]
    # '--config' flag can be ommited to use default settings
    $ astrocast -–config "/path/to/config" motion-correction --h5-loc "dataset/name" "/path/to/file"
\end{lstlisting}


