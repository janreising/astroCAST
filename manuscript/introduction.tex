%! suppress = LineBreak

Astrocytes exhibit calcium fluctuations that are spatially and temporally diverse, reflecting their integration of a
wide range of physiological signals\citep{semyanov_making_2020,smedler_frequency_2014}. Our protocol provides a
detailed guide for extracting these astrocytic calcium events from video recordings, performing clustering analysis,
and correlating astrocyte activity with other physiological signals\fref{1}. This analysis has been applied to
astrocytes in various settings, including acute slices, cell cultures, and in vivo recordings. Specifically, we
demonstrate the application of astroCAST, using a custom dataset from a postnatal transgenic mouse brainstem,
focusing on the \ac{preBötC}, as well as utilizing publicly available data.

\subsection{Requirements}
The initial step involves preparing the video recordings for analysis. Our protocol supports a range of file formats
, including .avi, .h5, .tiff (either single or multipage), and .czi, accommodating videos with interleaved recording
paradigms. To ensure reliable results, recordings should be captured at a frequency of 8Hz for 1-photon imaging and
1-2Hz for 2-photon imaging. This frequency selection is crucial for capturing events of the expected duration
effectively. It is imperative that the recordings specifically capture astrocytes labeled with calcium sensors,
either through transgenic models or viral vectors. While the use of calcium dyes is possible, their application may
not guarantee the exclusive detection of astrocytic events.

Regarding hardware requirements, our protocol is adaptable to a variety of configurations, from personal computers to
cloud infrastructures, with certain modules benefiting from GPU acceleration. At a minimum, we recommend using
hardware equipped with a 1.6GHz quad-core processor and 16GB of RAM to efficiently handle the data analysis.

On the software side, we advocate for the use of Linux operating systems, specifically Ubuntu or AlmaLinux, for
optimal performance, although Windows or macOS can be used with some functional limitations. Be advised, the M1 and
M2 Mac processors are currently not supported, and we recommend using the docker image instead. However, even using
the docker image it is not guaranteed that astroCAST will perform as expected. Essential software
includes Python version 3.9, along with Anaconda or Miniconda for managing Python environments, and git for version
control. An optional recommendation is ImageJ or an equivalent image viewer for analyzing the output visually. This
comprehensive approach ensures that researchers can accurately extract and analyze astrocytic calcium signals, paving
the way for further understanding of their physiological roles.